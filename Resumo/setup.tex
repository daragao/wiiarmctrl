%\documentclass[11pt,a4paper]{article}
\documentclass[conference]{IEEEtran}

% ======
% Lingua
% ======
\usepackage[latin1]{inputenc}

% ==================
% Pacotes auxiliares
% ==================
\usepackage[caption=false,font=footnotesize]{subfig}
\usepackage{multirow}
\usepackage{slashbox}
\usepackage{verbatim}
\usepackage{subfig}
\usepackage{acronym}
\usepackage[hyphens]{url}
\usepackage{amssymb}             % simbolos matematicos extra
%\usepackage{pstricks,pst-tree}   % graficos em Postscript
\usepackage[final]{graphicx}            % imagens
\usepackage{epstopdf}
\usepackage{multicol}            % multiplas colunas
\usepackage{listings}		 % para programas
\usepackage{microtype} % makes pdf look better
\lstset{language=C,
  basicstyle=\ttfamily,
  keywordstyle=\ttfamily\bfseries,
  ndkeywordstyle=\ttfamily,
  %indent=-5.5ex,
  %stringspaces=false,
}
\usepackage{pdfpages}
\usepackage[breaklinks,hyperindex=true, pdfnewwindow=true, pdftitle={Comparing new and old interfaces for the control of an humanoid robot},pdfsubject={Computer Engineering Master Thesis Dissertation}, pdfauthor={Duarte Aragao}, pdfkeywords={interaction, humanoid, comparison, Kinect sensor and Wii Remote}, colorlinks=false, pagebackref=false, citecolor=blue, plainpages=false, pdfpagelabels, pdfborder={0 0 0}]{hyperref}

% ======================================================
% Macros especificas da dissertacao, facilitam a escrita
% ======================================================
\def\imply{\rightarrow}

% ============
% Configura��o
% ============
% Dimensoes da folha A4paper = 210mm x 297mm
\setlength{\textheight}{23.3cm}
\setlength{\textwidth}{15.6cm}

% Topo
\setlength{\topmargin}{-\headheight}
\setlength{\headsep}{0.43cm}

% Lados
% 3cm + 2.5cm = 5.5cm,  210-55 = 155mm = \textwidth (com marginparsep=marginparwidth=0)
\setlength{\marginparwidth}{0cm}
\setlength{\marginparsep}{0cm}
\setlength{\oddsidemargin}{0.45cm}
\setlength{\evensidemargin}{0.46cm}

\renewcommand{\baselinestretch}{1.5}   % espacamento de 1.5 linhas
\setlength{\parindent}{0in}            % recorte nulo na 1� linha do paragrafo

\setcounter{secnumdepth}{3}     % m�xima profundidade para sec��es
\setcounter{tocdepth}{3}        % indice at� 3 niveis