
	The motivational question for this work was ``how can novel gaming interfaces compare as a humanoid interface?''. The answer was explored through an evaluation made where several interfaces were used with the same goal so that their performance could be compared.

	The strongest conclusion statement of this work is that familiar interfaces perform better for the untrained people. This could be taken from the fact that the most successful interface during the evaluation was the \ac{DPad} interface. Even with limited control as with this interface users were able to perform very well the tasks proposed. The great strength of this interface was that users knew what to expect, because they are used to have this kind of interface and could easily relate what they expected to what happened.
	
	The Kinect although a novel interface was the second most successful interface. This was the interface were users saw more possibilities and the easiest to learn. But being the interface that was the most unknown augmented the uncertainty among users that it would not really obey in the expected way, making users spend much time testing it out to understand what the iCub reaction would be. Being a common interface eliminates this skepticism making the users have a predefined idea of what might happen, as it was the case with the \ac{DPad} interface. Having a predefined idea helps to adapt to the real interaction.
	
	The \ac{Wiimote} interaction supports this idea because users that were not able to grasp the concept of the interface, performed poorly, while users that were able to understand it, slightly performed better. A user that during a test understood very well how the interaction worked was able to have a very impressive performance, although in the beginning of the test the user performance was very low. The \ac{Wiimote} was the less common control method, and also the most difficult to relate to. Those were considered the main reasons for the difficulty in understanding the interface.